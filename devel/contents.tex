\title{Project: Applying security and usability in practice}
\author{Daniel Bosk}

\maketitle

\mode*

%\begin{abstract}
%  Security, privacy and usability have all gained traction in recent years.
High usability has been a must-have since smartphones and tablets entered the 
scene as an alternative to the personal computer.
Privacy was probably best emphasized through the advent of \ac{GDPR} in 2018.
\Ac{GDPR} also implies strong security.
This means that, for any product to succeed, it must have a strong emphasis on 
usability, privacy and security.

This project aims for you to practice and show that you are able
\begin{itemize}
  \item to \emph{evaluate} the usability of security solutions and 
    \emph{suggest} improvements that improve usability and security.
  \item to \emph{evaluate} threats, possible protection mechanisms and to 
    \emph{design} a high-level approach to protection which considers 
    usability.
  \item to \emph{navigate} the field of information security, 
    \emph{distinguish} your own limits and where to search for solutions, \eg 
    experts or published research results that are relevant to the solution of 
    a given problem.
  \item to \emph{analyse and apply} the results of published research in the 
    security field.
\end{itemize}

%\end{abstract}

\section{Introduction}%
\label{sec:intro}

Security, privacy and usability have all gained traction in recent years.
High usability has been a must-have since smartphones and tablets entered the 
scene as an alternative to the personal computer.
Privacy was probably best emphasized through the advent of \ac{GDPR} in 2018.
\Ac{GDPR} also implies strong security.
This means that, for any product to succeed, it must have a strong emphasis on 
usability, privacy and security.

This project aims for you to practice and show that you are able
\begin{itemize}
  \item to \emph{evaluate} the usability of security solutions and 
    \emph{suggest} improvements that improve usability and security.
  \item to \emph{evaluate} threats, possible protection mechanisms and to 
    \emph{design} a high-level approach to protection which considers 
    usability.
  \item to \emph{navigate} the field of information security, 
    \emph{distinguish} your own limits and where to search for solutions, \eg 
    experts or published research results that are relevant to the solution of 
    a given problem.
  \item to \emph{analyse and apply} the results of published research in the 
    security field.
\end{itemize}



\section{Assignment}%
\label{sec:work}

Improve a current product or invent a new one.
The focus should be to use or incorporate research results from the area of 
information security.
\Eg it could be to improve privacy properties for streaming 
services~\cite[\eg][]{anonpass} or
improve usability and security for payment systems.

First look at the current usability and security in the area.
Look for published research that improves the situation.
Suggest improvements and methods to evaluate these.

You should summarize your work in a report.
First summarize how you performed your evaluation, what its results are.
Then summarize your suggestions with motivations why these improve the 
situation.

Based on your report, make a presentation that will be a research-founded 
\enquote{sales pitch} to your security-aware classmates.


\section{Examination}%
\label{sec:exam}

The report must be handed-in in the course platform in PDF or PostScript 
format, no other formats are accepted.
You can find the timeslots for presentations in the course schedule.

The report will be graded according to the following grading criteria:
\begin{description}
  \item[Grade E] You fulfil all the \acp{ILO} at the minimum level.
  \item[Grade A] You fulfil all the \acp{ILO},
    your evaluations and designs are extensive and well-founded in theory and, 
    where applicable, the research literature and otherwise show deep insights 
    in how to use the building-blocks from the literature.
\end{description}
The grades B, C and D are intermediary grades.
When assessing the multi-dimensional domain of your work, we will try to make 
as fair a projection as possible onto this linear scale.


\printbibliography
