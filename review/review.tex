\documentclass[a4paper]{llncs}
\usepackage[utf8]{inputenc}
\usepackage[swedish,english]{babel}
\usepackage[hyphens]{url}
\usepackage{hyperref}
\usepackage{cleveref}
\usepackage{listings}
\usepackage[natbib,style=numeric-comp,maxbibnames=99]{biblatex}
%\addbibresource{review.bib}

\title{Seminar: Current Research Literature in Security}
\author{%
  Daniel Bosk
}
\institute{%
  Department of Information and Communication Systems\\
  Mid Sweden University, Sundsvall
}

\begin{document}
\maketitle


\section{Introduction}%
\label{sec:Introduction}

This seminar concerns current research in the area of information security.
As many other research areas in technology, the area of security is an area 
which moves forward very quickly.
This makes it very important to be able to take part of new research in this 
area, otherwise your security level might decrease as the world around you 
moves forward.
Since the lack of security measures can be very uncomfortable, it is in 
everyone's best interest to keep up with this subject.


\section{Aim and Scope}%
\label{sec:Syfte}

The aim of this seminar is that you get a deepened overview of different areas 
within information security, and after it you should be able to:
\begin{itemize}
	% $Id$
% Author:	Daniel Bosk <daniel.bosk@miun.se>
\item Read and understand research results in the area of security.
\item Reflect on the relevance and research methodology used in studies.

\end{itemize}


\section{Theory}%
\label{sec:Theory}

% $Id$
For this assignment you should, with corroboration of the tutor, choose one to 
three research papers (depending on their size, they should be around 15 to 20 
pages in total) in information security.

Since the idea is to deepen your knowledge in some areas of information 
security, you must choose papers strongly related to the course.
For example, you can focus on areas such as:
\begin{itemize}
  \item usable security,
  \item privacy enhancing technologies (PETs), or
  \item more advanced methods for guessing passwords.
\end{itemize}
These are just examples, you are free to choose the area and papers in 
corroboration with the tutor.



\section{Assignment}

You are first going to read the papers you have chosen and maybe discuss the 
results with a colleague.
Then you should prepare a 10 minute long presentation of these papers, you will 
give this presentation to the class during the seminar as a base for 
discussion.
The presentation should present the following about the papers:
\begin{enumerate}
  \item research questions,
  \item research methodology,
  \item the results and their relevance for security.
\end{enumerate}
After your presentation we will discuss what you have presented to the class.


\section{Examination}%
\label{sec:Examination}

This assignment is examined by the presentation and the participation in the 
seminar.

The assignment may be done in groups of up to two individuals.
If you work in groups the examiner will choose one of the group members to do 
the whole presentation.


\printbibliography{}
\end{document}
