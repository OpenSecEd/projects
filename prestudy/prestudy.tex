\documentclass[a4paper]{article}
\usepackage[swedish,english]{babel}
\usepackage[utf8]{inputenc}
\usepackage[hyphens]{url}
\usepackage{hyperref}
\usepackage[strict]{csquotes}

\usepackage[all]{foreign}
\renewcommand{\foreignfullfont}{}
\renewcommand{\foreignabbrfont}{}

\usepackage[natbib,style=numeric-comp,maxbibnames=99]{biblatex}
\addbibresource{prestudy.bib}

\usepackage{beamerarticle}
\usepackage{cleveref}

\title{Project: A Short Study in Information Security}
\author{%
  Daniel Bosk
}
\institute{%
  Department of Information and Communication Systems\\
  Mid Sweden University, Sundsvall
}

\begin{document}
\maketitle


\section{Introduction}%
\label{sec:intro}

The area of information security paces forward at high speed due to the 
constant arms race between attackers and defenders.
It is thus very important to continuously review and improve one's security.
If you do not, then your adversary is the only one who will---and your 
adversary will not tell you about the findings.

There are also other reasons for having security.
For instance, when enormous amounts of information about people is collected 
into databases it is advisable to oversee the protection of that information, 
but perhaps more importantly, what can be done with it.
Firstly, it is important for the individuals whose information is stored in such 
a database; it is their information, they are affected if the information is 
leaked unintentionally (breach) or intentionally, \eg the organization owning 
the database changed their \enquote{terms of service} and started using the 
information in other ways---ways which might be unacceptable to the individual.
Secondly, it is important for the organization holding this information, 
specifically in the case of a breach.
% XXX add ref to Equifax


\section{Scope and Aim}%
\label{sec:goal}

The aims of this assignment is that you should deepen your knowledge of 
security by making a short study in one of the areas in security.

After completing this assignment you should be able to:
\begin{itemize}
	% $Id$
% Author:	Daniel Bosk <daniel.bosk@miun.se>
\item Read and understand research results in the area of security.
\item Reflect on the relevance and research methodology used in studies.

\end{itemize}


\section{Theory}

The base theory of this assignment is the foundational knowledge in security.
Then you will also need to deepen your knowledge by looking into current 
research literature.
Exactly what papers depends on the topic you choose and you choose them in 
corroboration with the tutor.


\section{Assignment}%
\label{sec:work}

You are going to make a short study within the area of information security.
The area is chosen in corroboration with the tutor.
The first thing you should do is to find a proper set of research questions.
For inspiration see the section \enquote{Research Problems} in the end of each 
chapter in Anderson's \citetitle{Anderson2008sea}~\cite{Anderson2008sea}.
Other ideas for this project are the following:
\begin{description}
  \item[Security in an organisation]
    Agree with the head of security in some organisation and analyse the 
    security in their organisation.
    This can be \eg performing a social-engineering-based attack, some form of 
    penetration testing, or performing a gap analysis.
    This is a prestudy, but the methodology should be very well worked out for 
    a larger follow-up study.
    (Do not forget the research connection, \eg methodology: why this study is 
    best designed this way.)

  \item[Advanced Persistent Threats]
    This would be a systematization of knowledge.
    Research documented APT events, summarize and provide an overview:
    what happens, to whom, by whom, why?
    What can we learn and why is it important?

  \item[Security usability]
    An example study could look into increasing the usability and security of 
    \eg payment systems.
    First look at the current usability and security, then suggest improvements 
    and methods to evaluate these.

  \item[Applied security]
    Improve a current product, or invent a new one, using research results from 
    the area of information security.
    \Eg better privacy properties in streaming services~\cite{anonpass}.
    Then your presentation will be a sales pitch to your security-aware 
    classmates.
    Your report should then be usable for selling this product to a company for 
    further development and production.

  \item[Reproducible research]
    Read and replicate some \enquote{cool paper}, \eg~\cite{acoustic}.
    Read how they did it, then you do the same setup and show it off to the 
    class.
    You should present your insights, e.g.:
    how difficult this is to perform,
    what is required to perform it.
    In summary, who is the adversary and what can he or she do with this.
\end{description}

Once you have chosen a problem to focus on you start your work.
Your results must be scientifically founded.
You have inspiration for your methodology from experiences from earlier 
courses, but also from the methods presented in the research papers.
Remember to think about what you want to show, how to show it and \emph{why} 
this method shows this correctly.

Start writing the report directly from the start, it is usually much easier to 
write the report while doing the work.
Start by writing the introduction with aims and what problem to solve.
Then continue with the section on methodology as you work on developing how to 
answer your questions.
After this you can go ahead and do the actual work, then write the results and 
analyse them.
Finally discuss and conclude your findings in the last section of the report.


\section{Examination}%
\label{sec:exam}

Your study should result in a written academic report and an oral presentation 
of said report.
The report must be handed-in in the course platform in PDF or PostScript 
format, no other formats are accepted.
You can find the timeslots for presentations in the course schedule.

The report (and presentation) must provide the reader with a short overview of 
the security field to illustrate wherein the treated topic belongs.
%(This does not mean that you should write ten pages summarizing the whole 
%course in detail.)

The assignment may be done in groups of up to two people.
However, if the project proposed is of proper size you may be excused from this 
limit.
Talk to the tutor and motivate well why you have to be a larger group.


\printbibliography{}
\end{document}
